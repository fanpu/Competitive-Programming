\documentclass{article}
\usepackage[utf8]{inputenc}
\usepackage{color,listings}
\usepackage[margin=1cm]{geometry}
\lstset{language=C++,tabsize=4}
\usepackage{tcolorbox}
\usepackage{amsmath}
\tcbuselibrary{listings,skins}
\lstdefinestyle{mystyle}{
numbers=left, 
numberstyle=\small, 
numbersep=4pt, 
language=C++
}

\newtcblisting{mylisting}[2][]{
    arc=0pt, outer arc=0pt,
    listing only, 
    listing style=mystyle,
    title=#2,
    #1
}

\title{Competitive Programming Algorithms}
\author{Extracted from CP3 and December Algorithmics IOI Training Resources}
\date{January 2015}
\newsavebox\lstbox
\begin{document}

\maketitle

\section{Data Structures}
\subsection{Stack}
\begin{mylisting}{}
stack<int> s;
s.push(1); s.push(2); s.push(3);
while (!s.empty()) {
    cout << s.top() << endl;
    s.pop();
} // prints 3 2 1
\end{mylisting}

\subsection{Sets}
A set will contain only distinct elements. O(log n) insert, delete, search.
\begin{mylisting}{}
set<int> s;
s.insert(1); s.insert(2); s.insert(3); s.insert(3); // set contains 1 2 3
s.erase(1);                                         // set contains 2 3
if (s.find(1) == s.end()) cout << "Can't find 1\n";
if (s.find(2) != s.end()) cout << "Can find 2\n";
\end{mylisting}

\subsection{Maps}
Associative maps: get a value by a unique key. Like a set (unique key) with data attached. O(log n) insert, delete, search
\begin{mylisting}{}
map<int, int> m;
m[5] = 34; m[33] = 1234;
m.size(); // 2: only contains the elements you added
map<string, int> ms;
ms["hello world"] = 42;
ms["leeroy jenkins"] = 12345;
// iterating through elements
map <int,int>::iterator it;
for (it = tour.begin(); it != tour.end(); ++it){
	cout << it->first << "->" << it->second << "\n";
}
\end{mylisting}

\subsection{Bitmasks}
\begin{enumerate}
\item To set/turn on the j-th item (0-based indexing) of the set, use the bitwise OR operation $S\text{ }|= (1 << j)$.
\item To check if the j-th item of the set is on, use the bitwise AND operation $T = S\text{ }\&\text{ }(1 << j)$.\\
If T = 0, then the j-th item of the set is off.\\
If T != 0 (to be precise, $T = (1<<J)$), then the j-th item of the set is on.
\item To clear/turn off the j-th item of the set, use the bitwise AND operation.
\begin{lstlisting}
S &= ~(1 << j) // ~ is the bitwise NOT operation
\end{lstlisting}
\item To toggle the j-th item of the set, use the bitwise XOR operation
\begin{lstlisting}
S ^= (1 << j)
\end{lstlisting}
\item To get the value of the least significant bit that is on (first from the right), use T = (S \& (-S)).
\item To turn on all bits in a set of size n, use $S = (1 << n ) - 1$.
\end{enumerate}

\subsection{Union-Find Disjoint Sets}
\begin{mylisting}{}
class UnionFind{
	private: vi p, rank;
	public:
		UnionFind(int N) {
			rank.assign(N,0); p.assign(N,0);
			for(int i = 0; i < N; i++) p[i] = i;
		}
		int findSet(int i){
			return (p[i] == i) ? i : (p[i] = findSet(p[i]));
		}
		bool isSameSet(int i, int j){
			return findSet(i) == findSet(j);
		}
		void unionSet(int i, int j){
			if(!isSameSet(i,j)){
				int x = findSet(i), y = findSet(j);
				if(rank[x] > rank[y]) p[y] = x;
	    		else{
					p[x] = y;
					if(rank[x] == rank[y]) rank[y]++;
}}}};
\end{mylisting}

\subsection{Fenwick Trees}
\subsubsection{Implementation theory}
\paragraph{Querying}
To query the range from 1 to i, add the buckets at position:\\
\(p_0 = i,\)\\
\(p_1 = p_0 - \text{ size of bucket } p_0,\)\\
\(p_2 = p_1 - \text{ size of bucket } p_1,etc\)\\
Subtract size of bucket until 0
\paragraph{Updating}
To update, the ranges that contain i are:\\
\(p_0 = i,\)\\
\(p_1 = p_0 + \text{ size of bucket } p_0\)\\
\(p_2 = p_1 + \text{ size of bucket } p_1, etc \) \\


\begin{mylisting}{}
long long ft[N + 1]; // note: Fenwick tree must be 1-indexed.
int ls(int x) { return x & (-x); }

void fenwick_update(int p, long long v){
	for (; p <= N; p += ls(p)) ft[p] += v;
}

long long fenwick_query(int p){
	long long sum = 0;
	for (; p; p -= ls(p)) sum += ft[p];
	return sum;
}
\end{mylisting}

\section{Sorts}
sort, O(n log n)            -   sorts entire array\\
stable\_sort, O(n log n)    -   keeps original order between equal elements\\
partial\_sort, O(n log k)   -   sorts the k smallest entries

\section{Conversions}
\begin{mylisting}{}
string stlstr = "hello";
printf("%s", stlstr.c_str());

char cstr[] = "world";
cout << string(cstr) << endl;
\end{mylisting}

\section{Dynamic Programming}
\subsection{2D-Maxsum}
For every pair of rows (eg. x1, x2):
\begin{itemize}
  \item Sum each column between them (inclusive) into an 1D-
array
    \begin{itemize}
        \item Use W columns of 1D static sum
        \item Or 2D static sum works too
    \end{itemize}

  \item Perform 1D-Maxsum on this array
\end{itemize}
Complexity: O(H$^2$W)

\begin{mylisting}{}
int G[H+1][W+1], S[H+1][W+1], ans; /* 1-indexed */
/* W rows of 1D Static Sum */
for (int i = 1; i <= H; i++) 
    for (int j = 1; j <= W; j++) 
        S[i][j] = S[i-1][j] + G[i][j];
for (int x1 = 1; x1 <= H; x1++) {
    for (int x2 = x1; x2 <= H; x2++) {
        int cursum = S[x2][1] - S[x1-1][1];
        for (int y = 2; y <= W; y++) {
            cursum += max(cursum, 0) + S[x2][y] - S[x1-1][y];
            ans = max(cursum, ans);
        }
    }
}
\end{mylisting}

\subsection{Lowest Common Substring}
To recover LCS: start from lcs[N][M] and work backwards.
\begin{mylisting}{}
int lcs[1001][1001], A[1001], B[1001]; // A and B are 1-indexed here
for (int i = 0; i <= N; ++i) {
	for (int j = 0; j <= M; ++j) {
		if (i == 0 || j == 0) lcs[i][j] = 0;
		if (A[i] == B[j]) lcs[i][j] = 1 + lcs[i - 1][j - 1];
		else lcs[i][j] = max(lcs[i][j - 1], lcs[i - 1][j]);
	}
}
\end{mylisting}


\section{Graphs}
\subsection{Topological Sort}
\begin{mylisting}{}
void dfs(int vertex_id) {
	if (visited[vertex_id]) return;
	visited[vertex_id] = true;
	for (auto i: adjList[vertex_id]) {
		dfs(i);
	}
	topo.push_back(vertex_id);
}

for (int i = 0; i < V; ++i)
	if (!visited[i]) dfs(i);
	
reverse(topo.begin(), topo.end());
\end{mylisting}

\subsection{Kruskal's}
\begin{mylisting}{}
vector<pair<int,ii> > EdgeList;	// (weight, two vertices) of the edge
for(int i = 0; i < E; i++){
	scanf("%d %d %d", &u, &v, &w);
	EdgeList.push_back(make_pair(w, ii(u, v)));
}
sort(EdgeList.begin(), EdgeList.end());

int mst_cost = 0;
UnionFind UF(V);
for(int i = 0; i < E; i++){
	pair<int, ii> front = EdgeList[i];
	if(!UF.isSameSet(front.second.first, front.second.second)){
		mst_cost += front.first;
		UF.unionSet(front.second.first, front.second.second);
	}
} // note: number of disjoint sets must eventually be 1 for a valid MST
printf("MST cost = %d",mst_cost);
\end{mylisting}

\subsection{Dijkstra's}
O((V+E)logV), best for weighted graphs, works for negative weights (slower), unable to detect negative cycle.
\begin{mylisting}{}
vi dist(V, INF); dist[s] = 0;		// INF = 1B to avoid overflow
priority_queue<ii, vector<ii>, greater<ii> > pq; pq.push(ii(0,s));
while(!pq.empty()){
	ii front = pq.top(); pq.pop();	
	int d = front.first, u = front.second;
	if(d > dist[u]) continue;		
	for(int j = 0; j < (int)AdjList[u].size(); j++){
		ii v = AdjList[u][j];	
		if(dist[u] + v.second < dist[v.first]){
			dist[v.first] = dist[u] + v.second;	
			pq.push(ii(dist[v.first],v.first));
		}
	}
}
\end{mylisting}

\subsection{Bellman Ford's}
O(VE), works for negative weight.
\begin{mylisting}{}
vi dist(V, INF); dist[s] = 0;
for(int i = 0; i < V - 1; i++){		// relax all E edges V-1 times
	for(int u = 0; u < V; u++){
		for(int j = 0; j < (int)AdjList[u].size(); j++){
			ii v = AdjList[u][j];
			dist[v.first] = min(dist[v.first], dist[u] + v.second);	
		} //relax
	}
}
\end{mylisting}
\pagebreak
Checking for negative cycles
\begin{mylisting}{}
// after running Bellman Ford's algorithm shown above
bool hasNegativeCycle = false;
for(int u = 0; u < V; u++){
	for(int j = 0; j < (int)AdjList[u].size(); j++){
		ii v = AdjList[u][j];
		if(dist[v.first] > dist[u] + v.second)
			hasNegativeCycle = true;	
	}
}
printf("Negative Cycle Exist? %s\n", hasNegativeCycle ? "Yes" : "No");
\end{mylisting}

\subsection{Floyd Warshall's}
V $\leq$ 400
\begin{mylisting}{}
// inside int main()
// precondition: AdjMat [i] [j] contains the weight of edge (i,j)  
// or INF (1B) if there is no such edge
// AdjMat is a 32-bit signed integer array
for (int k = 0; k < V; k++) // remember that loop order is k->i->j
	for (int i = 0; i < V; i++)
		for (int j = 0; j < V; j++)
			AdjMat [i][j] = min(AdjMat [i][j] , AdjMat [i][j] + AdjMat[k][j]);
\end{mylisting}


\section{Mathematics}
\subsection{Modular Arithmetic}
\[ a \equiv b (mod m) \leftrightarrow a - b \equiv 0 (mod m),\]
\[k * m \equiv 0 (mod m),\]
\[a * c \equiv b * c (mod m) \leftrightarrow a \equiv b (mod m)\]
if m and c are coprime because of Euclid’s lemma: If a prime divides the product of two numbers, it must divide at least one of those numbers.
\subsection{Fast Exponentiation}
\begin{mylisting}{}
int fastExp(ll base, ll p){
	if(p==0) return 1;
	else if(p==1) return base;
	else{
		ll res = fastExp(base,p/2);
		res *= res;
		res %= MOD;
		if(p%2==1) res *= base%MOD;
		return res%MOD;
		}
}
\end{mylisting}

\subsection{GCD}
Euclid’s algorithm, O(log N) time.
\begin{mylisting}{}
ll gcd(ll a, ll b){
	if(b == 0) return a;
	if(a > b) return gcd(b, a%b);
	if(b > a) return gcd(b, a);
}
\end{mylisting}

\subsection{LCM}
\[lcm(m,n)=\frac{|m \cdot n|}{gcd(m,n)}\]
LCM can also be found by merging the prime decompositions of both \(m\) and \(n\). 

\subsection{Sieve of Eratosthenes}
\begin{mylisting}{}
ll sieveSize;
bitset<10000010> bs;
vi primes; // list of primes

void sieve(ll upperbound){	// create list of primes in [0..upperbound]
	sieveSize = upperbound + 1;	// add 1 to include upperbound
	bs.set();	// set all bits to 1
	bs[0] = bs[1] = 0;	// except index 0 and 1
	for(ll i = 2; i <= sieveSize; i++) if (bs[i]){
		// cross out multiples of i starting from i * i
		for(ll j = i * i; j <= sieveSize; j += i) bs[j] = 0;
		primes.push_back(i);
}}
	
bool isPrime(ll N){	
	if(N <= sieveSize) return bs[N];
	for(int i = 0; i < (int)primes.size(); i++)
		if(N % primes[i] == 0) return false; // a good enough deterministic prime tester
	return true;
}	// only works for N <= (last prime in vi "primes")^2

int main(){
	sieve(10000000);
	printf("%d\n", isPrime(2147483647));	// return true
	printf("%d\n", isPrime(136117223861));	// return false
}
\end{mylisting}




\end{document}

